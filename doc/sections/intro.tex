\section{مقدمه}
برنامه ریزی مسیر یکی از مهمترین پایه های ربات های محرک خودمختار است که به ربات اجازه می دهد که کوتاهترین یا بهینه ترین مسیر بین دو نقطه پیدا کند. مسیر بهینه می تواند مسیری باشد که میزان تعداد چرخش، ترمز کردن یا هر چیز بخصوص که مد نظر است کمینه باشد. الگوریتم ها برای پیدا کردن کوتاهترین مسیر نه تنها در رباتیکز، بلکه در مسیریابی شبکه ها، بازی ویدیویی و … اهمیت دارند.

در برنامه ریزی مسیر، یک نقشه از محیط و آگاهی ربات از موقعیت خود نسبت به نقشه نیاز است. در این گزارش این فرض ها رو داریم:

\begin{itemize}
    \item ربات توانایی موقعیت یابی خود را دارد
    \item ربات مجهز به نقشه است
    \item ربات توانایی دوری از مانع های را دارد
\end{itemize}
اینکه چگونه نقشه را بدست می آوریم یا خود را موقعیت یابی می کنیم و یا چگونه با عدم قطعیت اطلاعات موقعیت عمل می کنیم در این گزارش پوشش داده نمی شوند.

