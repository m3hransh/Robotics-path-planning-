\documentclass{article}
\usepackage{mystyle}

%%title Page
\title{برنامه ریزی مسیر}
\titleextra{
 (برنامه ریزی با اطلاعات کامل، فضای موقعیت، گراف دیداری)}
\author{ محمد مهران شهیدی}
\authortwo{محمد مهرداد شهیدی}
\authorthree{سروش مهدی}
\supervisor{جواد افشاری}
\date{\today}
%%title Page

\begin{document}
\maketitle
\pagenumbering{alph}
\begin{abstract}
برنامه ریزی مسیر یکی از مسائل مهم در بحث رباتیک و ربات های محرک است. مسئله این است که چگونه یک ربات میتواند از نقطه ایی  شروع به یک نقطه دلخواه برسد و مجموعه حرکت را چگونه بدست بیاوریم. در این گزارش چگونگی نشان دادن موقعیت نقاط و موانع بحث می شود و متناسب با آنها الگوریتم هایی بحث خواهد شد. و برخی از این الگوریتم ها نیز در زبان پایتون پیاده و نتیجه با هم مقایسه شده اند. در این گزارش در مورد  راه حل هایی برای یکسری مشکلات در برنامه ریزی به خصوص مشکلات محاسباتی که ممکن است رخ دهد بحث شده. و در آخر به برخی از کاربرد های دیگر برنامه ریزی مسیر پرداختیم.
برای دسترسی به کد مربوط به پیاده سازی ها از لینک زیر استفاده کنید.
\newline
\begin{latin}
\href{https://github.com/m3hransh/Robotics-path-planning-}{https://github.com/m3hransh/Robotics-path-planning-}
\end{latin}
\end{abstract}
\indent\indent
\newpage
\tableofcontents
\newpage
\pagenumbering{arabic}
\import{sections/}{intro}
\import{sections/}{sec1}
\import{sections/}{sec2}
\import{sections/}{sec3}
\import{sections/}{sec4}
\import{sections/}{sec5}
\import{sections/}{sec6}
\begin{thebibliography}{9}
\bibitem{latexcompanion}
\begin{latin} 
Nikolaus Correll 
\textit{Introduction to Autonomous Robots}. 
Magellan Scientific, March 6, 2020.
\end{latin} 


\end{thebibliography}

\end{document}
\end{document}
